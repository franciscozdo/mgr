\chapter{Conclusion}
\label{chapter:conclusion}

In this thesis I have presented the design of Virtual Memory subsystem in a UNIX system.
In particular we have seen the details of the part responsible for managing the processes memory.
Moreover, we have seen the real implementation of the UVM virtual memory subsystem with the copy-on-write mechanism in the Mimiker OS.
All changes to Mimiker, descibed in this thesis, are alread merged to the {\tt master} branch in the project repository, hence every user of the Mimiker OS is now using the new implementation of Virtual Memory.

\section{Contributions}

In previous chapters we have seen the new features added to the Mimiker while working on this thesis.
The main goal was the effective implementation of copy-on-write mechanism.
The one created there has improved the performance of the system as expected.
The number of page copies has been reduced and the kernel spends less time during a {\tt fork} system call thanks to that improvement.

Additionally, in order to make the process of implementing copy-on-write mechanism easier, I have reworked the implementation of the virtual memory map structures.
Their new design, based on the UVM, is more modular, because of different structures used to describe different types of memory.
It not only made the implementation of copy-on-write simpler, but will also help with implementing other features of virtual memory subsystem in the future.

Other changes, also in subsystems different from the VM, has improved the overall experience while developing Mimiker.
The new information associated with signals has created new possibilities while designing user space tests that weren't possible before.
In particular, the tests used to verify correctness of the VM subsystem has been improved the most.
They were extensively used while working on new VM design.

The last contribution to the Mimiker OS is the Kernel Function Trace utility.
It was used to check if the implementation introduced earlier behaves as expected.
Additionally, it helped with finding optimizations in the implementation of the virtual memory subsystem.
For example, the \mintinline{c}{pmap_protect} function was optimized to operate well on large memory regions
(before it was implemented to work effectively on on single page regions).

Because the KFT is split into two independent parts: kernel part and library for analysis,
it may be used in the future for similar performance analysis for other parts of the kernel.
Because the Python library translates KFT dump into format that is easy to manage within Python,
it can be used to implement any kind of analysis on the data gathered during Mimiker run.

\section{Future work}

As mentioned earlier, the new design will allow for gradual and simpler implementation of next features of the VM subsystem.
Two of them, memory mapped files and paging, are the most important ones.
Memory mapped files will use the UVM~objects as a structure to describe the memory that will reflect the contents of a file.
However, they require also some work in the scope of Virtual File System in order to manage the files that are accessed via file system.
The VFS implementation in Mimiker needs to be revisited and improved to make it possible to interact with other subsystems.

The paging mechanism is also the one that should be possible to add in the future.
It will implement the interaction between virtual memory and devices used to store data.
The infrastructure needed to operate on block devices, that are used to store blocks of data, needs to be created in Mimiker.
Once the paging mechanism is implemented, it will allow to swap out pages from main memory to the backing storage.

There are also other features that can be later added, but are not essential in the Mimiker at the moment.
For instance, the additional UVM mechanisms that were mentioned in one of the previous chapters (page loan, page transfer and map entry passing).
They are not essential, because they solve problems that aren't currently experienced in the Mimiker, because they are used during operations that
are impossible or not extensively used in the Mimiker (e.g. fetching large chunks of data from the internet) .
