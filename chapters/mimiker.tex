\chapter{Implementation of UVM in Mimiker}
\label{chapter:mimiker}

The practical part of this thesis is the implementation of the UVM-like virtual memory subsystem in Mimiker.
In this chapter, we will see the motivation behind this change, an overview of the system that has been implemented,
and some of the related work that has been done besides the actual implementation.

\section{The Mimiker Operating System}

Mimiker \cite{mimiker:website} is a UNIX operating system intended for research and teaching.
Its implementation is inspired by the systems from the BSD family.
Its design focuses on simplicity, as the system is used to teach students about the internals and design of the UNIX kernels.
Mimiker is still under development so there are many features that are gradually added into the kernel (one of such features is a copy-on-write support).

The Mimiker repository \cite{mimiker:sources} is used to store all the code and configuration needed to build, run, and test Mimiker.
The contents of the repository are:
\begin{itemize}
  \item Kernel code.
  \item Testing infrastructure -- both kernel and userspace tests.
  \item Userspace programs -- programs ported from the BSD systems to provide some basic user experience when launching Mimiker's shell.
  \item Scripts used to run Mimiker in Qemu emulation \cite{qemu:website}.
  \item Toolchain -- configuration and custom patches needed to prepare toolchain required to compile and simulate Mimiker.
  \item Basic documentation.
\end{itemize}

The Mimiker OS is currently ported to three different architectures: MIPS (on~Malta board), ARM (on Raspberry Pi 3) and RISCV (on HiFive Unleashed).
All these architectures support virtual memory and implement hierarchical page tables (but each of them defines different number of levels in the page table).
Later we will not talk much about the details of these architectures,
because they are not important to understand the implementation of virtual memory and copy-on-write mechanism.

\section{Motivation}

The Virtual Memory subsystem used in Mimiker was based on old BSD~VM system.
The implementation wasn't complete, there were many features missing, in particular: copy-on-write support, memory mapped files and memory swapping.
Some of such features are not implemented because there is no infrastructure to implement them
(e.g. no support for block devices or unstable VFS implementation).

Copy-on-write functionality doesn't require any additional subsystem to be enhanced beforehand.
However, after evaluating the implementation of it in current VM subsystem we have discovered that it might quickly get complicated
even if many features of virtual memory are missing.
To reduce the complexity of implemented VM system we have decided to investigate how UVM is built and replace the old VM system with it.

The integration of UVM simplified the implementation of VM system in Mimiker without losing any features.
We have made an observation that only the part responsible for anonymous memory must be implemented, because there are no other memory types supported yet.
(The first attempt to implement UVM in Mimiker failed without realizing this fact.)
We also hope that the design of UVM will help to gradually implement other VM features easily.

% \begin{itemize}
%   \item currently designed with old BSD~VM, but haven't implemented much features so far
%   \item the implementation has no CoW support
%   \item no shadow objects, no memory mapped files, no backing storage (e.g. no support for block devices)
%   \item not much to change to switch to UVM
%   \item UVM is simpler and should be easy to improve over time (with more features like mm files etc.)
%   \item Concentrate more on things that got better (not the time of operations)!
% \end{itemize}

%\comment{I want this chapter to be a guide for someone who wants to understand the Mimiker's VM design.}

\section{Implementation of UVM in Mimiker}

The current implementation of UVM in Mimiker isn't complete, because the old implementation that was replaced with UVM wasn't complete either.
However, in contrast to the previous VM subsystem, the new one will be easier to extend with new features.
Currently, only the anonymous memory mappings are available, so only the infrastructure to manage them is created so far.
The two most important features of VM system that are missing are: paging and mapping files into memory.
To get the first one we will need to extend the current implementation with support for backing storage, and to make it possible
to create mappings that reflect files memory we will need to implement UVM~objects.
Both these features will not require to change the implementation of the amaps and will require to slightly modify implementation of
anons to allow for assigning the swap number to the anon's page.
The other, more advanced features of UVM (e.g. page loanout and page transfer) are hard to evaluate now,
but they will be build upon the current implementation of amaps and anons.

\subsection{Structures}

To describe the VM~map of the process we use standard \mintinline{c}{vm_map} and \mintinline{c}{vm_map_entry} structures.
The only difference from UVM is that \mintinline{c}{vm_map_entry} structure has no reverence to the UVM~object, but only to the amap.
Both these structures, presented on listings \ref{impl:vm_map} and \ref{impl:vm_map_entry}, are defined in \mintinline{c}{sys/kern/vm_map.c} \cite{mimiker:sources}.

\begin{listing}[h]
  \begin{minted}{c}
struct vm_map {
  TAILQ_HEAD(vm_map_list, vm_map_entry) entries;
  size_t nentries;
  pmap_t *pmap;
  mtx_t mtx;
};
  \end{minted}
  \caption{VM~map structure}
  \label{impl:vm_map}
\end{listing}

\begin{listing}[h]
  \begin{minted}{c}
struct vm_map_entry {
  TAILQ_ENTRY(vm_map_entry) link;
  vm_entry_flags_t flags;
  vm_aref_t aref;
  vm_prot_t prot;
  vaddr_t start;
  vaddr_t end;
};
  \end{minted}
  \caption{VM~map entry structure}
  \label{impl:vm_map_entry}
\end{listing}

Amaps used in Mimiker are implemented using arrays.
Whey they are created, a fixed size array is allocated to be used for anons storage.
Such decision was made to make the implementation as simple as possible.
The performance problem that might occur in an array implementation is not really the issue here,
because the memory segments that appear in Mimiker are not too big and their amaps can quickly be traversed sequentially.

Amaps are referenced from \mintinline{c}{vm_map_entry} structure using a small structure aref in the same way as in UVM.
It is used to store an offset into amap, which makes it easier to split map entries without splitting or copying the amap.
The amap structure, which is presented on listing~\ref{impl:vm_amap}, is defined in \mintinline{c}{sys/kern/vm_amap.c} and the aref,
which is identical with the structure used in UVM, structure in \\ \mintinline{c}{include/sys/vm_amap.h} \cite{mimiker:sources}.

\begin{listing}[h]
  \begin{minted}{c}
  struct vm_amap {
    mtx_t mtx;             /* Amap lock. */
    size_t slots;          /* Maximum number of slots. */
    refcnt_t ref_cnt;      /* Number map entries using amap. */
    vm_anon_t **anon_list; /* Pointers of used anons. */
  };
  \end{minted}
  \caption{Amap structure}
  \label{impl:vm_amap}
\end{listing}

The other difference between UVM and its Mimiker implementation is the anon structure.
The page referenced by the anon structure can't be swapped out.
This means that the page reference will not change during the lifetime of anon, so the page for each anon can be allocated when the anon is created.
Later in this chapter we will sometimes identify anons wit pages (e.g. saying that the page is searched in the amap), because they are interrelated
and knowing a page there is exactly one anon that is referencing it.

The functions of anon could have been implemented by adding a reference counter to the \mintinline{c}{vm_page}.
However, such implementation would complicate the semantics of pages, and make it less clear how it is used.
Moreover, we hope that the UVM~objects will be implemented soon, and than having a separate structure for describing anons is mandatory.

Another difference in \mintinline{c}{vm_anon} structure, compared to the UVM, is the lack of anon lock.
Currently there is no need for such lock, because the only field that would be protected by it is the reference counter, which is currently atomic.
The anon structure, presented on listing~\ref{impl:vm_anon}, is implemented in \mintinline{c}{include/sys/vm_amap.h} \cite{mimiker:sources}.

\begin{listing}[h]
  \begin{minted}{c}
  struct vm_anon {
    refcnt_t ref_cnt; /* Number of amaps referencing this anon. */
    vm_page_t *page;  /* Page bound to the anon. */
  };
  \end{minted}
  \caption{Anon structure}
  \label{impl:vm_anon}
\end{listing}

\subsection{Operations on VM~map}

In the \mintinline{c}{sys/kern/vm_map.c} file there are defined all functions that are used by the kernel to manipulate the VM~map of the process.
Thanks to them, the implementations of individual system calls is simpler, and independent from the design of virtual memory.
These operations are also used to modify the memory map of the process in response to actions that are not directly meant to modify memory
(e.g. all operations on processes like \mintinline{c}{fork}, \mintinline{c}{execve} and \mintinline{c}{exit}).

The most important functions:

\begin{description}[style=nextline]
  \item[{\tt void vm\_map\_new(void);}]
    Initialize the new empty VM~map.
    This operation is mainly used during {\tt fork} and {\tt exec}.

  \item[{\tt int vm\_map\_insert(vm\_map\_t *map, vm\_map\_entry\_t *entry, vm\_flags\_t flags);}]
    Insert new VM~map entry into existing VM~map when it is created (e.g. when {\tt mmap} system call is used).
    By default, the VM~map entry is inserted at any address where it could fit.
    The {\tt flags} argument may be used to change that behavior.
    The flags that are passed to this function come from the {\tt mmap} argumets:
    {\tt VM\_FIXED} -- specifies that the VM~map entry must be inserted at specified addresses, and
    {\tt VM\_EXCL} -- which additionally will force unmapping of any other memory mappings that overlaps with inserted one.

  \item[{\tt int vm\_map\_entry\_resize(vm\_map\_t *map, vm\_map\_entry\_t *ent, \\vaddr\_t new\_end);}]
    This function can either extend the address range covered by given map entry or reduce it.
    The VM~map entry can be extended when the underlying amap is big enough
    (currently amaps cannot be extended, but they are created with few extra slots that can be used when entry is extended).
    When the map entry is shrunk, the pages that were previously used to store data must be unmapped.
    This function is used, for example, when the program break is moved by {\tt brk} and {\tt sbrk} system calls.

  \item[{\tt vm\_map\_entry\_t *vm\_map\_find\_entry(vm\_map\_t *vm\_map, vaddr\_t vaddr);}]
    Function that determines the VM~map entry which describes given address.
    When no map entry is describing it, {\tt NULL} value is returned.

    The lookup of the VM~map entry is very handful operation and is used by almost all functions that implement system calls actions.
    When the requested operation is served there is usually a need to find the VM~map entry specified by the address given as an argument to the system call.
    (E.g. when removing the mapping, we have to first find the VM~map entry that is representing the memory that will be removed.)

  \item[{\tt int vm\_map\_protect(vm\_map\_t *map, vaddr\_t start, vaddr\_t end, \\vm\_prot\_t prot);}]
    Change access protection for specified address range.
    If part of VM~map entry must be changed, for instance because the start address is inside the memory segment,
    then the VM~map entry is split into two individual entries, and the protection is changed only for the affected entry.
    After updating information about access protection, it must be also updated in page table.
    This function is used when the protection change is requested by the {\tt mprotect} call,
    or when the new segments with the specified protection are created during the creation of a new process during {\tt exec}.

  \item[{\tt int vm\_map\_destroy\_range(vm\_map\_t *map, vaddr\_t start, vaddr\_t end);}]
    Remove all memory segments within specified range when it is requested by {\tt munmpa} system call.
    Update page table of current process to remove address mappings that won't be valid after call to this function.
    Similarly to the previous function, if only part of VM~map entry must be destroyed then it has to be split into two parts.

  \item[{\tt void vm\_map\_delete(vm\_map\_t *vm\_map);}]
    Destroy entire VM~map of the process after it has ended its life (e.g. after calling {\tt exit} or being terminated as a result of unhandled signal).

  \item[{\tt vm\_map\_t *vm\_map\_clone(vm\_map\_t *map);}]
    Create a new VM~map during the {\tt fork}.
    Created VM~map is identical to the original one and shares all the VM~map entries with the old one.
    This function also takes care about all flags that must be set to handle properly copy-on-write entries.

  \item[{\tt int vm\_page\_fault(vm\_map\_t *map, vaddr\_t fault\_addr, \\vm\_prot\_t fault\_type);}]
    Function invoked by machine dependent code, after page fault exception was generated by MMU.
    It is responsible for finding the page that the process tried to access and inserting it into the process's page table.
    If this is impossible, because the page doesn't exit or has a different access protection than expected,
    it must return one of {\tt EACCES} or {\tt ENOMEM} to indicate the cause of the page fault.

\end{description}

The last two functions, \mintinline{c}{vm_map_clone} and \mintinline{c}{vm_page_fault}, are more complicated
and important to understand the implementation of copy-on-write.
They will be discussed in more details in following sections.

\subsection{Operations on amaps and anons}

In order to implement functions described above we need to operate on amaps and anons referenced from VM~map entries.
All essential functions used to manage amaps and anons are defined in \mintinline{c}{sys/kern/vm_amap.c}.

The most important functions:

\begin{description}[style=nextline]
  \item[{\tt vm\_aref\_t vm\_amap\_copy\_if\_needed(vm\_aref\_t aref, size\_t slots);}]
    Make copy of amap only when it is needed.
    This function is executed after {\tt VM\_ENT\_NEEDSCOPY} flag was discovered in VM~map entry holding that amap.
    Amap is actually copied when it has more than one reference.
    Such copy is done during page fault, when the new anon must be inserted into given amap.
    If amap is still shared between multiple VM~map entries, then it is automatically copied using this function to
    safely insert new anons into it.

  \item[{\tt void vm\_amap\_remove\_pages(vm\_aref\_t aref, size\_t offset, \\ size\_t n\_slots);}]
    Remove anons from specified slots in the amap (e.g. when memory segment is destroyed).
    If removed anons were referenced only by the current amap, they are freed after being removed from the amap.
    If they are referenced also by another amap, their reference count is decreased by 1.

  \item[{\tt void vm\_amap\_insert\_anon(vm\_aref\_t aref, vm\_anon\_t *anon, \\ size\_t offset);}]
    Save anon at specified offset in the amap.
    This function fails when we try to insert anon into slot where other anon is already stored.
    New anons are inserted into the amap when new anonymous memory is created or when they are copied after being shared as copy-on-write anons between multiple amaps.

  \item[{\tt vm\_anon\_t *vm\_amap\_find\_anon(vm\_aref\_t aref, size\_t offset);}]
    Check if amap has an anon at given offset and return its structure.

  \item[{\tt vm\_anon\_t *vm\_anon\_copy(vm\_anon\_t *src);}]
    Make a copy of given anon.
    The new anon is allocated and the page referenced from the original one is copied to the new one.
    Copy of the anon is needed, when anon is referenced by multiple anons but it is representing private memory.
    After copying anon, each amap that were referencing it will have its own copy of it.

\end{description}

\subsection{Cloning process VM~map}

The \mintinline{c}{vm_map_clone} function is responsible for making a copy of VM~map for the child process created during {\tt fork} operation.
Each VM~map entry that is a part of VM~map must have its copy in the new map.
Hence memory segments might be private or shared between processes, there are different rules that define how to clone them.

Whole function is organized into single loop over all VM~map entries of the old VM~map.
Each VM~map entry is examined to check its type, and using this type, proper copying function is selected.
There are two different functions:

\begin{description}[style=nextline]
  \item[\mintinline{c}{vm_map_entry_clone_shared(vm_map_t *map, vm_map_entry_t *ent);}]
    This function is used to copy a VM~map entry that describes shared memory.
    New VM~map entry is identical to the original one.
    The reference count of the amap used by both entries must be increased to record that.
    If an amap hasn't been allocated yet, it is the last time when it can be done.
    (If we don't allocate amap now, then none of the processes that is using these VM~map entries would be capable of allocating common amap.)

  \item[\mintinline{c}{vm_map_entry_clone_copy(vm_map_t *map, vm_map_entry_t *ent);}]
    This function is used to copy a VM~map entry that describes private memory.
    In this case, the amap is not copied either, but both new and original VM~map entry must be updated.
    In both entries the \mintinline{c}{VM_ENT_COW} and \mintinline{c}{VM_ENT_NEEDSCOPY} are added to flags to indicate that they represent copy-on-write segment
    and that the amap must be copied before new pages are inserted to it.
    The reference count in the amap is increased, because both entries will be using it unless it is copied.

    Additionally, page table of the parent process must be updated, to specify that the pages of the copied segment are now read-only.
    Proper access protection will be restored when they will be actually copied after first write fault.
\end{description}

After the new, copied VM~map entry is obtained, it is inserted into new VM~map.
Whole \mintinline{c}{vm_map_clone} function is presented on listing~\ref{impl:vm_map_clone}.

\begin{listing}[h]
  \begin{minted}{c}
vm_map_t *vm_map_clone(vm_map_t *map) {
  /* ... */
  vm_map_t *new_map = vm_map_new();
  vm_map_entry_t *it, *new;
  TAILQ_FOREACH (it, &map->entries, link) {
    switch (it->flags & VM_ENT_INHERIT_MASK) {
      case VM_ENT_SHARED:
        new = vm_map_entry_clone_shared(map, it);
        break;
      case VM_ENT_PRIVATE:
        new = vm_map_entry_clone_copy(map, it);
        break;
      default:
        panic("Unrecognized vm_map_entry inheritance flag: %d",
              it->flags & VM_ENT_INHERIT_MASK);
    }
    /* ... */
    TAILQ_INSERT_TAIL(&new_map->entries, new, link);
  }
  return new_map;
}
  \end{minted}
  \caption{The essential part of \mintinline{c}{vm_map_clone}}
  \label{impl:vm_map_clone}
\end{listing}

%* operation made during fork \\
%* create new vm map for the child process \\
%* each vm map entry must be either copied to new vm map or the reference is duplicated to share the entre between processes \\
%
%* the main part of the copying funciton is the loop traversing all entries of the original vm map \\
%* it is responsible for creating new vm map entry by proper function (selected based on inheritance flags) \\
%* the new entry is then inserted into the new vm map \\
%
%* this is the place where the implementation changed a lot \\
%* the vm map clone funciton is now much tidier and easier to  modify \\
%* after changing this implementation, the copy-on-write was just a matter of modifying map cone copy function \\

\subsection{Page fault handling}

Second important function of VM subsystem is handling page faults.
In Mimiker it is implemented in the \mintinline{c}{vm_page_fault} function.

The operations needed to be done during page fault are different for different types of faults.
At the beginning of the page fault routine we have to determine which scenario do we are in.
Two main scenarios are copy-on-write and non copy-on-write fault.

The copy-on-write fault happens when process is trying to write to memory which is marked as copy-on-write and still shared between two processes.
In such case, the memory must be copied.

In both cases, kernels searches for the VM~map entry that is describing the memory which is accessed by the process.
If such map entry isn't found, the access is not valid and kernel will send {\tt SIGSEGV} signal to the user process with info that
this is a map error (\mintinline{c}{SEGV_MAPERR}).
When the VM~map entry is found, then it is used to search for the page that contains faulting address.

%* another important operation in VM subsystem \\
%* invoked by machine dependent code to check if memory access that failed should in fact be successful
%
%* heavily depends on the implementation of storage for pages \\
%* how to determine cow or not cow case

\subsubsection{Standard fault}

The non-copy-on-write fault is easier to handle and is done with fewer steps:

\begin{enumerate}
  \item If amap reference doesn't exist in the VM~map entry, the amap must be created, because new page will be created and inserted into it.
  \item Page is searched in the amap (using \mintinline{c}{vm_amap_find_anon}).
  \item If page is not found, then it must be created.
  \item The found or new page must be inserted into the amap and page table.
    \begin{itemize}
      \item If VM~map entry has \mintinline{c}{VM_ENT_COW} flag set and represents segment of memory that can be written by the process,
        then the write access permission to the inserted page must be removed, to trigger copying of the page before first write to it.
    \end{itemize}
\end{enumerate}

%* amap is not present -> create \\
%* page is not found -> create \\
%* page is found -> insert \\

\subsubsection{Copy-on-write fault}

The copy-on-write fault is more complicated, and there are few things kernel must carefully do.

\begin{enumerate}
  \item If VM~map entry doesn't have an amap then create one, because we will be inserting a new anon into it.
  \item Check if amap is shared between multiple VM~map entries.
    Use \\ \mintinline{c}{vm_amap_copy_if_needed()} to obtain new \mintinline{c}{vm_aref} that will point to the correct amap
    (when amap is exclusive for the current VM~map entry, then the returned aref is identical to original one).
  \item Search for the anon in the amap.
  \item Clone the anon if it needs to be cloned.
    \begin{itemize}
      \item Check if anon's reference count is greater than 1 (if that's true, then the anon must be copied).
      \item Create a new anon with a copy of the page hold by the original anon.
      \item Replace the old anon in the amap with the reference to new one.
    \end{itemize}
  \item Insert new page table entry that describes the new page (using \mintinline{c}{pmap_enter}).
\end{enumerate}


\section{The UVM implementation process in Mimiker}

Implementing new features in the existing software is always a complicated task.
Many different parts of the codebase interact with each other and these interactions are sometimes hard to spot.
Especially when working with a huge codebase, which is the case of working on a operating system.
In Mimiker, it is common to modify many different subsystems while working on a single feature.
This is because there are many things to improve, and new or improved features help find bugs in many parts of the operating system.

\subsection{Changes prior to the UVM implementation}

Prior to the actual change in the virtual memory implementation, a few improvements were required in the old VM subsystem and other subsystems.
All these changes are aimed at improving the test infrastructure and the quality of tests of the virtual memory subsystem.
These new and modified tests helped a lot later, during the transition to the UVM.
All improvements made before this transition were implemented in the part of code that haven't been modified when implementing the UVM system.
The implementation of system calls and virtual memory tests don't require the knowledge of the structures used to store individual pages.
Because of that, the internal representation of structures which describe memory segments can be changed.

\subsubsection{New and improved system calls}

To express more operations on process memory, I had to review existing system calls to see if all the nuances of their semantics were correctly implemented.
During that process the {\tt mmap} and {\tt munmap} were improved.
{\tt mmap} system call was extended to support \mintinline{c}{VM_FIXED} and \mintinline{c}{VM_EXCL} flags.
In order to do that, the function for destroying the memory range has to be updated, to make it possible to use it in that case.
In the situation where both of these flags are set during a {\tt mmap} call, any memory that conflicts with the created memory mapping must be removed.
This change also affected the {\tt munmap} system call, because the same procedure is used to destroy memory region specified by this system call.
These changes were also aimed at being compliant with POSIX \cite{posix} standard.

The other important system call that operates on process memory is the {\tt mprotect} call.
The mechanism responsible for protecting memory from some accesses wasn't implemented at all.
Even the memory created during the {\tt exec} call wasn't protected (allowing the process to write to all memory segments, even the ones that should be read-only).

The ability to restrict memory accesses to parts of memory has been added gradually.
At first, the basic functionality was implemented in the \mintinline{c}{vm_map_protect} to allow for changing protection of whole memory segments.
At the beginning it was only used during the creation of new memory map for the process during the {\tt exec} syscall.

The next natural step was to implement a fully functional {\tt mprotect} call that would allow memory protection to be changed at any time.
The implementation of \mintinline{c}{vm_map_protect} was extended to support changing the protection of any portion of the valid memory range.
When only a part of \mintinline{c}{vm_map_entry} needs to have the access protection changed, then the entry is split into multiple parts.
The function fails, if some memory within specified range is not mapped.

After improving the system calls or creating the new ones, the tests of virtual memory must have been updated.
I have created new test cases for all new possible actions.

\begin{samepage}
Below I show pull requests that were created during that step:
\begin{itemize}
  \item \pr{1345}{[execve] Set protection bits during ELF load}
  \item \pr{1357}{Improve memory operations}
  \item \pr{1370}{Implement mprotect syscall}
  \item \pr{1383}{vm: Fix prot check in vm\_page\_fault}
\end{itemize}
\end{samepage}

\subsubsection{Improvements in signal infrastructure}

The signaling infrastructure is important for the VM memory management.
It provides a mechanism to inform the process about invalid memory operations.
The signals were already implemented, but they only provided information about the signal number and nothing more.
To provide more information, it was necessary to extend the implementation of the \mintinline{c}{sig_trap} function.
This function is responsible for recording information about the signal that will later be delivered to the process.

To provide additional information bound to the signal, the \mintinline{text}{ksiginfo_t} structure is filled in.
The structure and all functions needed to operate on it, were already implemented, but it was only used to store information about the signal number.
The important thing was to provide the additional information that will later be recorded in the \mintinline{text}{ksiginfo_t} structure.
In the case of signals used to communicate the exception caught by the kernel to the process, this information is generated in the machine-specific trap handler code.
Besides the {\tt SIGSEGV}, there are also other signals generated there ({\tt SIGBUS}, {\tt SIGFPE}, {\tt SIGILL} and {\tt SIGTRAP}).
Along with all these signals, the kernel stores more detailed information about the exception that occurred in the \mintinline{c}{ksiginfo_t} structure,
which is later translated into \mintinline{text}{siginfo_t} to be passed to the userspace signal handler.

The signals mentioned above fill in the \mintinline{text}{siginfo_t} fields with following information:
\begin{itemize}
  \item \mintinline{c}{si_addr} -- address of memory location or the instruction that triggered the signal
  \item \mintinline{c}{si_code} -- detailed information about the type of operation that caused the fault
\end{itemize}

In case of {\tt SIGSEGV} the \mintinline{c}{si_addr} points to the address that was accessed by the process and the \mintinline{c}{si_code}
tells the type of memory fault: \mintinline{c}{MAP_ERROR} -- accessed memory was unmapped, \mintinline{c}{ACC_ERR} -- memory was protected from requested access.

These improvements were introduced in pull request: \pr{1363}{Save additional information to signals}.

\subsubsection{Improvements in testing infrastructure}

The testing infrastructure is an important part of the system implementation.
It allows to quickly discover errors in newly implemented features or in the redesigned code.
With new actions available through system calls and more detailed information in signals, it was possible to implement new tests and improve the old ones.

The most important improvement was the introduction of macros that help to define parts of tests that expect to receive a signal due to a forbidden action.
Before that, it was really hard to create a test that checks if an action will fail with a given signal.
It was also impossible to test more than one such action per test, because the signal was checked after the whole test failed.

By introducing a new method of validating signals, new tests have become possible.
Now we can check if a fragment of code causes a signal, and verify that all the information delivered with it matches our expectations.
I have created a macro \mintinline{c}{EXPECT_SIGNAL} which defines which signal is expected to be received while executing instructions in the next code block.
Under the hood, this macro sets up a signal handler to record if the signal was caught, and if caught,
then saves the \mintinline{text}{siginfo_t} structure associated with it.
Immediately after using the \mintinline{c}{EXPECT_SIGNAL}, the signal handler used internally by it, must be cleared using \mintinline{c}{CLEANUP_SIGNAL}.

After successfully capturing the signal, the recorded \mintinline{text}{siginfo_t} structure should be tested for expected values.
Each signal has its own macro defined, to check appropriate fields in that structure.
In the case of {\tt SIGSEGV} the \mintinline{c}{CHECK_SIGSEGV} macro is used.
All these macros are described in the \mintinline{c}{bin/utest/util.h} header in the source code \cite{mimiker:sources}.

That new construct was added into testing infrastructure in \\
\pr{1364}{Use sigaction in all tests that need to handle SIGSEGV}.

On listings \ref{impl:segv_accerr_test} and \ref{impl:segv_maperr_test} there are two code snippets,
that test if the code inside the \mintinline{c}{EXPECT_SIGNAL} block causes the \mintinline{c}{SIGSEGV} with correct
\mintinline{c}{si_addr} and \mintinline{c}{si_code} recorded in the \mintinline{text}{siginfo_t} structure.

\begin{listing}[h]
  \begin{minted}{c}
siginfo_t si;
EXPECT_SIGNAL(SIGSEGV, &si) {
  /* The memory pointed by ptr is read protected.
   * This operation should raise SIGSEGV. */
  (void)(*ptr == 0);
}
CLEANUP_SIGNAL();
CHECK_SIGSEGV(&si, ptr, SEGV_ACCERR);
  \end{minted}
  \caption{Testing if memory read is truly forbidden.}
  \label{impl:segv_accerr_test}
\end{listing}

\begin{listing}[h]
  \begin{minted}{c}
siginfo_t si;
EXPECT_SIGNAL(SIGSEGV, &si) {
  /* The memory at addr + 0x2000 is not mapped.
   * This operation should raise SIGSEGV. */
  int data = *((volatile int *)(addr + 0x2000));
  (void)data;
}
CLEANUP_SIGNAL();
CHECK_SIGSEGV(&si, addr + 0x2000, SEGV_MAPERR);
  \end{minted}
  \caption{Test with access to unmapped memory}
  \label{impl:segv_maperr_test}
\end{listing}

\subsection{Transition to UVM}

After all preliminary work has been done we were able to change the implementation of the VM subsystem.
In fact the change was intended to be as small as possible and to simplify the whole VM implementation a lot.
It was achieved by removing two old structures (VM objects and pagers) and introducing one new structure (VM amap).
It was possible, because VM objects and pagers were used only to represent anonymous memory mappings without any additional features.
Thanks to that all required operations made on memory segments were possible to implement with just an amap.
Moreover, there was no need to implement anons at that moment, because without copy-on-write support all memory pages were eagerly copied during the fork,
so each page could be owned only by one amap at the time.
Because there wasn't introduced any new functionality, the new VM subsystem could be tested with the old tests.

The transition to UVM based virtual memory was made in \\ \pr{1379}{[vm] Simplify virtual memory subsystem}.

\subsection{Implementation of copy-on-write}

When the new VM subsystem was ready, we could finally start implementing the copy-on-write mechanism.
Thanks to the previous improvements in the VM subsystem the change was easier to create.
In this stage, the implementation of amaps was extended to make it possible to represent pages with anons which could shared between multiple amaps.

The changes that were made to create copy-on-write mechanism were done in two places: in {\tt fork} implementation and in page fault routine.
Changes made to the \mintinline{c}{vm_map_clone} function, invoked during {\tt fork}, were rather simple.
The old, eager copying of the whole VM~map was removed and instead only the flags of the VM~map entries are modified.
Whole memory is shared right after the child process is created and the private memory segments are marked as copy-on-write to be copied later.

More complicated changes were done in the page fault handler implementation in the \mintinline{c}{vm_page_fault} function.
Because now this function has to determine if the memory has to be copied, it has to handle a few new cases.
It has to carefully check the flags and reference counters of the amaps and anons to determine if they have to be copied.
Moreover, it must also adjust the permissions that are used when the found page is inserted into the process page table.

All the steps that are done in the \mintinline{c}{vm_page_fault} function:

\begin{enumerate}
  \item Create the amap if current VM~map entry doesn't have one.
  \item Search for anon in the current amap.
  \item If entry is marked as copy-on-write (call \mintinline{c}{cow_page_fault}):
    \begin{enumerate}
      \item Check if amap needs to be copied and copy it
        (it is done when \\ \mintinline{c}{VM_ENT_NEEDSCOPY} is set and amap is referenced by more than one VM~map entry).
      \item Check if anon needs to be copied and copy it (it is done when anon is referenced by multiple amaps).
    \end{enumerate}
  \item Insert anon with new (or found) page into current amap (if it isn't already there).
  \item Insert the page referenced by the found anon into the page table with correct access protection.
\end{enumerate}


%* changes in two places:
%  * fork -- rather simple change. removed copying in favour of setting few flags
%  * page fault -- some new, complicated logic
%    * few new cases (memory could be shared, partially copied or entirely copied)
%    * all of these cases must be properly identified and handled
%    * changed how amaps are copied

When working on that change, also the old tests were used unmodified because from process perspective nothing should change.
The internal design of the VM should not affect how it is exposed to user processes.

% * also the old tests were used, because processes should not notice any change, because that is only internal representation of

Copy-on-write was implemented in \pr{1392}{vm: Copy on write}.

\subsection{Further work}

The implementation described in this chapter is not complete.
Some of the features of UVM that are supported in the NetBSD system are not yet created in Mimiker.
To support them other subsystems in Mimiker must be reviewed and improved.
For instance, to implement memory mapped files we would have to improve the VFS (Virtual File System) because the main structure defined there,
\mintinline{c}{vnode_t}, will be used by UVM~objects to fetch data from the file system and place it in the process memory.
Similarly, to allow pages to be swapped to secondary storage,
we would need to ensure that we could easily interact with block devices and use them to store pages that are not currently in use.
